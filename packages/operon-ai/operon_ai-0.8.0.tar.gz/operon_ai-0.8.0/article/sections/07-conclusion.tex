\section{Conclusion}

The transition from ``Prompt Engineering'' to ``Agentic Engineering'' represents a shift from alchemy to
chemistry. However, current methodologies remain fragile, relying on trial-and-error prompting rather than
rigorous architectural principles.

In this paper, we have demonstrated that Gene Regulatory Networks (GRNs) provide a proven architectural blueprint
for distributed, stochastic information processing. By formalizing this analogy through Applied Category Theory,
we have derived a suite of robust design patterns:
\begin{enumerate}[leftmargin=*]
\item \textbf{Robustness:} The use of Quorum Sensing and CFFL topologies to filter stochastic noise.
\item \textbf{Validity:} The use of Chaperone Proteins to enforce structural determinism on probabilistic outputs.
\item \textbf{Security:} The identification of Prion-like prompt injections and the topological defenses required
to stop them.
\item \textbf{Evolution:} The mapping of Horizontal Gene Transfer to dynamic tool loading and Endosymbiosis to
neuro-symbolic integration.
\end{enumerate}

We conclude that the future of reliable AI agents lies in biomimetic topology. By structuring our software
according to the logic of life---from the metabolic constraints of Ischemia to the symbiotic integration of code
and intuition---we inherit the billions of years of R\&D that biology has invested in solving the problem of
autonomous control.
